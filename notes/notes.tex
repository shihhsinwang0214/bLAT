\documentclass[12pt]{article}
\usepackage[margin=1.25in]{geometry}

\usepackage{calc}
\usepackage{amsmath}
\usepackage{amssymb}
\usepackage{multicol}
\newcommand{\bogus}[1]{}

\usepackage[hidelinks]{hyperref}
\usepackage{tikz}
\usetikzlibrary{arrows,positioning,calc}

% title format and spacing
\usepackage{titlesec}
\titleformat{\section}{\normalsize\bfseries}{\thesection}{1em}{}
\titleformat{\subsection}{\normalsize\bfseries}{\thesubsection}{1em}{}
\titleformat{\subsubsection}{\small\itshape}{\thesubsubsection}{1em}{}
\titleformat{\paragraph}[runin]
{\normalfont\small\itshape}{\theparagraph}{1em}{}
\titleformat{\subparagraph}[runin]
{\normalfont\small\itshape}{\thesubparagraph}{1em}{}
\titlespacing*{\section} {0pt}{2ex}{0.5ex}
% \titlespacing*{\bibliography} {0pt}{2ex}{0.5ex}
\titlespacing*{\subsection} {0pt}{1.5ex}{0.5ex}
% \titlespacing*{\paragraph} {0pt}{*0}{1em}
% %\titlespacing*{\paragraph} {0pt}{3.25ex plus 1ex minus .2ex}{1em}
% \titlespacing*{\subparagraph} {\parindent}{*0}{1em}
% %\titlespacing*{\subparagraph} {\parindent}{3.25ex plus 1ex minus .2ex}{1em}

\input texFiles/defs

\usepackage{amsthm}
\newtheorem{definition}{Definition}
\newtheorem{theorem}{Theorem}
\begin{document}


\section{bLAT Layer}
{\definition 
The bi-Lipschitz affine transformation (bLAT) layer 
is given by the transformation, $f:\mathbb{R}^{N}\rightarrow\mathbb{R}^{M}$,
defined by
\begin{align}
f(x) = U \Sigma V^{\rm T} x + b,
\end{align}
%
where 
$U\in\mathbb{R}^{M\times r}$,
$V\in\mathbb{R}^{r\times N}$,
$\Sigma\in\mathbb{R}^{r\times r}$,
$b\in\mathbb{R}^{M}$,
$r = \min(M, N)$,
$U$ and $V$ are orthogonal matrices satisfying 
$U^{\rm T} U = V^{\rm T} V = I$,
and 
\begin{align}
    \Sigma = {\rm diag}
    \left[\begin{array}{ccc} \sigma_1 & \dots & \sigma_r\end{array}\right],
\end{align}
%
where $\frac{1}{L} \le \sigma_i \le L$ for 
$i=1,\dots, r$ and some $L\ge 1$.
When $M=N$, the mapping is invertible.\label{def:blat}
\begin{align}
    f^{-1}(x) = V \Sigma^{-1} U^{\rm T} (x - b).
\end{align}
}

\medskip

\noindent Our new contribution is to include the
matrix term $U\Sigma V^T$, which is written as a 
singular value decomposition (SVD),
which exists for every matrix in $\mathbb{R}^{M\times N}$.
One advantage of the bLAT layer is the following theorem.

{\theorem The forward and 
inverse bLAT layers are both $L$-bi-Lipschitz.
}

\medskip

\noindent This follows from the fact that
$\frac{1}{L} \le \|\grad f(x)\|_2 \le L$ and 
$\frac{1}{L} \le \|\grad f^{-1}(x)\|_2 \le L$.
Another advantage of using the SVD is that,
by construction, the evaluation of 
$f(x)$ and $f^{-1}(x)$ have an equivalent computational cost.
In our implementation of the $M=N$ case,
the orthogonal matrices of the 
bLAT layers are parameterized using the matrix exponential 
of a skew-symmetric matrix input,
i.e. $U = {\rm expm}(S)$, where $S=-S^T$.
When $M\ne N$, the orthogonal matrices are parameterized
using the Householder factorization.

\section{Orthogonal/Unitary Parameterizations}

\subsection{Matrix Exponential}

Let $S\in\mathbb{C}^{M\times M}$ be a skew symmetric matrix ($S=-S^T$), then
\begin{align}
  U = {\rm expm}(S)
\end{align}
%
is unitary. (Note: when $S\in\mathbb{R}^{M\times M}$, then $U$ is orthogonal.)

\subsection{Householder}

The notes in this section were originally written to help with implementation of code.
Consider an orthogonal matrix $Q\in\mathbb{R}^{n\times n}$. The Householder decomposition is defined by
\begin{align}
Q &= Q_n \dots Q_1, \\
Q^T &= Q_1^T \dots Q_n^T = Q_1 \dots Q_n,
\end{align}
%
Where $Q_1, \dots, Q_n$ are symmetric elementary orthogonal matrices defined by
\begin{align}
Q_k = \left[\begin{array}{cc}
I_{N-k \times N-k}&0\\0&F_k
\end{array}\right], \qquad
F_k = I_{k\times k} - 2 \frac{v_k v_k^T}{v_k^T v_k},
\qquad
v_k\in\mathbb{R}^{k}.
\end{align}
%
We can also define $Q_k$ using
\begin{align}
    Q_k = \tilde{F}_k = I_{N\times N} 
    - 2 \frac{\tilde{v}_k \tilde{v}_k^T}{\tilde{v}_k^T \tilde{v}_k},
    \qquad
    \tilde{v}_k = \left[\begin{array}{c}0\\v_k\end{array}\right] \in \mathbb{R}^N.
\end{align}
%
Note that $Q_k$ is symmetric.
\begin{align}
    \tilde{F}_k^T = I_{N\times N} 
    - 2 \frac{\tilde{v}_k \tilde{v}_k^T}{\tilde{v}_k^T \tilde{v}_k}
\end{align}
%
Matrix action of $Q_k$:
\begin{align}
    Q_k y =  y
    - 2 \frac{\tilde{v}_k \tilde{v}_k^T}{\tilde{v}_k^T \tilde{v}_k} y,
\end{align}
%
Consider $y\in\mathbb{R}^M$ $Q\in\mathbb{R}^{M\times N}$,
$\tilde{v}_k\in\mathbb{R}^N$, then
\begin{align}
    y^T Q_k =  y^T
    - 2 \frac{ \tilde{v}_k^T}{\tilde{v}_k^T \tilde{v}_k} (y^T \tilde{v}_k).
\end{align}
%

\subsection{Polcari}

Orthogonal matrices can be parameterized using the
Polcari decomposition, which was originally introduced in~\cite{polcari}.
Given an orthogonal matrix, $Q\in\mathbb{R}^{N\times N}$,
then $Q$ can be decomposed as
\begin{align}
  Q = \Psi(w_{N-1}) \hdots \Psi(w_1) S(\varphi_1, \dots, \varphi_N),
  \label{eq:polcarisigned}
\end{align}
where $w_j \in \{x: x\in \mathbb{R}^{j},\, \|x\|_2 \le 1\}$, for $j=1,\dots,N-1$,
and $\varphi_1, \dots, \varphi_N \in \{1, -1 \}$, and 
\begin{align*}
  &\Psi(w_j) = 
  \left[\begin{array}{ccc}
          {\rm I}_{j} - \frac{w_j w_j^{\rm T}}{1 + \sqrt{1 - w_j^{\rm T} w_j}} & w_j & \\
          - w_j^{\rm T} & \sqrt{1 - w_j^{\rm T} w_j} & \\
                                                             && {\rm I}_{N-j-1}
        \end{array}\right], \qquad i=1, \dots, N-1, \\
  &S(\varphi_1, \dots, \varphi_N) = \left[\begin{array}{ccc}
                                            \varphi_1 \\
                                           & \ddots \\
                                           && \varphi_N
                                         \end{array}\right].
\end{align*}
%
Here, ${\rm I}_d$ represents a $d \times d$ identity matrix.
Note that a vector, $v_j \in \mathbb{R}^i$, can be projected inside the unit ball,
$w_j \in \{x: x\in \mathbb{R}^{j},\, \|x\|_2 \le 1\}$, using a transformation such as
\begin{align*}
  w_j = \rho_j \frac{v_j}{\|v_j\|}, \qquad \rho_j = {\rm tanh}(\|v_j\|) \in [0, 1].
\end{align*}
%

The decomposition in~\eqref{eq:polcarisigned} can be used as a parameterization
to express any orthogonal matrix. Given an orthogonal matrix, $Q\in\mathbb{R}^{N\times N}$,
the parameters of the Polcari decomposition can be uniquely determined.
Consider the following properties.
\begin{enumerate}
\item $\Psi(w_j)$ for $j=1,\dots, N-1$ and $S$ are orthogonal matrices and therefore, by construction,
  the right-hand-side of~\eqref{eq:polcarisigned} is orthogonal. 
\item $\Psi(w_j) {\rm e}_k = {\rm e}_k$ for $k>j+1$,
  where ${\rm e}_k$ is a vector with a $1$ in the $k^{\rm th}$ component and $0$'s elsewhere.
\item ${\rm P}_{(k-1) N}\Psi(w_{k-1}) {\rm e}_{k} = w_{k-1}$, for $k=2,\dots N$,
  where ${\rm P}_{(k-1) N}$ is an identity matrix of size $(k-1) \times N$.
\item ${\rm e}_{k}^T\Psi(w_{k-1}) {\rm e}_{k} = \sqrt{1-w_{k-1}^{\rm T} w_{k-1}}$, for $k=2,\dots N$.
  % \item If $U$ is unitary and $u_i \defeq U {\rm e}_i$ is the $i^{\rm th}$ column of $U$, $u_i^{\rm T}u_i = 1$.
\item ${\rm P}_{(k-1)N} \Psi(w_{k-1}) \dots \Psi(w_1) S {\rm e}_k = \varphi_k w_{k-1}$,
  for $k=2,\dots N$. This follows from (2) and (3).
\item ${\rm e}_{k}^T \Psi(w_{k-1}) \dots \Psi(w_1) S {\rm e}_k = \varphi_k \sqrt{1 - w_{k-1}^{\rm T} w_{k-1}}$
  for $k=2,\dots N$. This follows from (2) and (4).
\end{enumerate}
%
% The parameters, $w_1, \dots w_{N-1}$ and $\varphi_1, \dots \varphi_N$,
% in~\eqref{eq:polcarisigned} can be determined uniquely to represent
% an arbitrary unitary matrix, $Q \in \mathbb{R}^{N \times N}$.
% Let the $j^{\rm}$ column vector be denoted by $q_j$ and let the element in row $i$ and column $j$ be denoted as
% $q_{ij}$.
Using properties (5) and (6) for $k=N$, we have
\begin{align*}
  P_{N-1,N}Q {\rm e}_N = \varphi_{N} w_{N-1}, \qquad {\rm e}_N^{\rm T} Q {\rm e}_N = \varphi_N \sqrt{1 - w_{N-1}^{\rm T} w_{N-1}},
\end{align*}
%
and therefore 
\begin{align*}
  w_{N-1} = \varphi_N P_{N-1,N}Q {\rm e}_N, \qquad \varphi_N = {\rm sign}\left({\rm e}_N^{\rm T} Q {\rm e}_N \right).
\end{align*}
%
Suppose that $w_{k}, \dots w_{N-1}$ is known. We can obtain relations for
$w_{k-1}$ and $\varphi_k$ using properties (5) and (6).
\begin{align*}
  {\rm P}_{(k-1)N} \Psi(w_{k})^{\rm T} \dots \Psi(w_{N-1})^{\rm T} Q {\rm e}_k
  % = {\rm P}_{(k-1)N} \Psi(w_{k-1}) \dots \Psi(w_1) S {\rm e}_k
  &= \varphi_k w_{k-1}, \\
  {\rm e}_k^{\rm T} \Psi(w_{k})^{\rm T} \dots \Psi(w_{N-1})^{\rm T} Q {\rm e}_k
  % = {\rm e}_k^{\rm T} \Psi(w_{k-1}) \dots \Psi(w_1) S {\rm e}_k
  &= \varphi_k \sqrt{1 - w_{k-1}^{\rm T} w_{k-1}}.
\end{align*}
%
Using this, general relations for $w_k$ and $\varphi_k$ for all $k$ can be obtained.
\bse
\begin{alignat}{2}
  w_{k} &= \varphi_{k+1} {\rm P}_{k,N} \Psi(w_{k+1})^{\rm T} \dots \Psi(w_{N-1})^{\rm T} Q {\rm e}_{k+1},
            \qquad &&k=1, \dots, N-1,\\
  \varphi_k &= {\rm sign}\left({\rm e}_k^{\rm T} \Psi(w_{k})^{\rm T} \dots \Psi(w_{N-1})^{\rm T} Q {\rm e}_k\right) ,
              \qquad &&k=1, \dots, N.
\end{alignat}\label{eq:polcariDetermine}
\ese
%
An algorithm for determining the Polcari parameters would use~\eqref{eq:polcariDetermine}
recursively, decrementing $k$ from $N$ to $1$.
Since the Polcari parameters can be uniquely determined, the following theorem can be stated.

{\theorem The decomposition given by~\eqref{eq:polcarisigned}
  is a parameterization for any orthogonal matrix in $\mathbb{R}^{N \times N}$.
  %This mapping involves $\frac{1}{2} N(N+1)$ real parameters.
  % \das{which is precisely the number of parameters
  % required to represent an arbitrary unitary matrix}.
}

\medskip

Note that since $S(\varphi_1, \dots, \varphi_N) U S(\varphi_1, \dots, \varphi_N)^T = U$,
$A(\theta_A)$ is invariant to the choice of $\varphi_1, \dots, \varphi_N$.
For simplicity, we choose $\varphi_1 = \dots = \varphi_N = 1$ in our parameterization,
since $S(1, \dots, 1) = I$.

\bibliographystyle{plain}
\bibliography{bib}


\end{document}